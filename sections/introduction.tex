\chapter{Introduction}
\label{cha:1_introduction} 
\parskip=0ex


\section{Motivation}
\label{sec:2_motivation}
According to World Health Organization (WHO)~\cite{deafnessAndHearingLoss}, there are over 5\% or about 466 million of world's population who have suffered of hearing disability. This number consists of 432 million adults and 34 million children. It is also predicted that by 2050, there will be over 900 million people or one in every ten people that will have the hearing disability.

A person can be said to have a hearing loss if he or she is not able to hear with a normal hearing thresholds of 25dB or better in both ears. The hearing loss may be divided into mild, moderate, severe or profound category. 'Hard of hearing' term refers to the individuals who have hearing loss ranging from mild to severe category. These people usually still can communicate using the spoken language and can get the benefit of hearing aids, cochlear implants and other devices. 'Deaf' refers to the people who mostly have profound hearing loss, which indicates the hearing ability is very little or no hearing. 

Sign Language is a language developed as a primary language by the Deaf community. These people often use Sign Language to communicate with each other. Although it is used as the primary language for the Deaf, it also can be a way to communicate for the Hard of Hearing, hearing individuals which are unable to physically speak, the people who have trouble with spolen language due to a disability or condition, and the Deaf family.

Similar to spoken language, different country has different Sign Language grammar and lexicon~\cite{WendySandler:2006}. This fact shows that sign language is unique in each world's region, not universal and not mutually understandable.

It has been proved that the reading ability of the deaf high school students is equal to the non-deaf students which are seven years younger. The median reading skills of Deaf students between 8 and 18 years old is equal to the non-deaf students in their fourth grade. The lack of phonemic awareness does have a big impact to their reading fluency. The Deafs are at a disadvantage compared to the hearing individuals because they cannot implicitly learn the relationship between letters and sounds without direct instruction and access to sound.

American Sign Language (ASL) is a natural language which dominates sign language of Deaf communities in the United States and most of Anglophone Canada. Not only North America, the dialects of ASL and ASL-based creoles are also used in many countries, including much West Africa and parts of South East Asia. ASL have some phonemic components, including the movements of the face and torso, together with the hands.

Based on U.S. Bureau of Labor Statistics~\cite{bureauOfLaborStat}, the median annual wage for interpreters and translators was \$49,930 in May 2018. The growth of interpreters and translators for Deaf is projected to be 18 percent from 2016 to 2026. Meanwhile the average for all occupation is just around 7 percent. So, it means although the wage of an interpreter is quite high, the demands of them is definetely growing from time to time.

Regarding the importance of ASL for Deaf, Athitsos et al.~\cite{ASLLexiconVideoDataset} made the availability of ASL Lexicon Video Dataset, a public dataset containing video sequences of thousands of distinct ASL signs, along with the annotations of those sequences. This dataset was made of purpose to provide a sign language dictionary for the hearing people and it hopes to be able to provide a baseline for sign language recognition. In this research, we want to try to make use of this dataset to provide a video-based interpreter for the Deaf.

The mentioned ASL dataset is a big collection of Sign Language videos presented by several native signers. A video contains gestures of several words based on Gallaudet Dictionary of American Sign Language. So, for each gesture to be presented in a video, the signer has to look into the guide and remember the gesture, and demonstrate it later. While the signer was looking to the guide, the camera was also still recording. Because of this reason, before and after a word was demonstrated, the signer will always stay in his/her resting position.

Detecting human body skeleton is a challenging research that has been done by various researchers. Many methods and implementations were proposed, including a method named Part Affinity Fields. This approach uses a nonparametric representation to learn to associate body parts with individuals in the image. An open library called OpenPose has made an implementation of it. The library can help us to detect human body skeleton from a 2D image or a 2D video in a real time manner.

In this research, we try to develop a transition motion selection algorithm to join some words from the ASL dataset into a video. We try to connect the gestures from the words to remove the signer's resting pose by finding the best candidates of frames collection and put them in-between to make the movement looks like they are continuous. We utilize the OpenPose algorithm to help us retrieve the similar frames.

\section{Contribution}
\label{sec:2_contribution}
In this research, we develop a system that can automatically generated a video-based human interpreter for the Deaf. In our system, the user has to input some gloss and define the threshold to the system. The contributions of this research are as follow:\begin{enumerate}
	\item This research introduces a new idea of using The American Sign Language Lexicon Video Dataset~\cite{ASLLexiconVideoDataset} to provide a video-based human interpreter for the Deaf.
	\item An approach to calculate similarity using the help of 2D skeleton detection library is implemented to support the proposed algorithm for the frame selection.
	\item The quality of the proposed method is evaluated by a user study, participated by native ASL signers and a smoothness evaluation method is also presented.
\end{enumerate}

\section{Outline}
\label{sec:3_outline}
The remainder of this thesis is organized as follow: introduction to ASL, previous text-to-ASL works, ASL Lexicon dataset, OpenPose Library and smoothness evaluation method are described in Chapter~\ref{cha:2_literature}. The proposed system including the system architecture design and the methodology that are used in this research are explained in Chapter~\ref{cha:3_design}. The results of the experiments conducted are discussed in Chapter~\ref{cha:4_prototypeImplementation}.  Finally, this research is concluded in Chapter~\ref{cha:5_analysis}.