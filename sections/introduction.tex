\chapter{Introduction}
\label{cha:1_introduction} 
\parskip=0ex

According to World Health Organization (WHO)~\cite{deafnessAndHearingLoss}, there are over 5\% or about 466 million of world's population have suffered of hearing disability. This number consists of 432 million adults and 34 million children. It is also predicted that by 2050, there will be over 900 million people or one in every ten people will have the hearing disability.

A person can be said to have a hearing loss if he or she is not able to hear with a normal hearing thresholds of 25dB or better in both ears. The hearing loss may be divided into mild, moderate, severe or profound category. 'Hard of hearing' term refers to the individuals who have hearing loss ranging from mild to severe category. These people usually still can communicate using the spoken language and can get the benefit of hearing aids, cochlear implants and other devices. 'Deaf' refers to the people who mostly have profound hearing loss, which indicates very little or no hearing. These people often use Sign Language to communicate. Although Sign Language is used as the primary language for the Deafs, it also can be a way to communicate for: the Hard of Hearing, hearing individuals which unable to physically speak, the people who have trouble with spoken language due to a disability or condition, and the Deaf family members.

Sign languages are the natural languages developed by each Deaf community all over the world with their own grammar and lexicon, which means each community has their own grammar and lexicon~\cite{WendySandler:2006}. This means that sign languages are unique for each region, not universal and not mutually intelligible.

American Sign Language (ASL) is a natural language which dominates sign language of Deaf communities in the United States and most of Anglophone Canada. Not only North America, the dialects of ASL and ASL-based creoles are also used in many countries, including much West Africa and parts of South East Asia. ASL have some phonemic components, including movement of the face and torso together with the hands.

It has been proved that the reading ability of the deaf high school students are equal to the hearing students which are seven years younger. The median reading skills of Deaf students between 8 and 18 years old are equal to the hearing students in their fourth grade. The lack of phonemic awareness does have a big impact to their reading fluency. The Deafs are at a disadvantage compared to the hearing individuals because they cannot implicitly learn the relationship between letters and sounds without direct instruction and access to sound. The vocabulary 

Regarding the importance of ASL for Deaf, Athitsos et al.~\cite{ASLLexiconVideoDataset} made the availability of ASL Lexicon Video Dataset, a public dataset containing video sequences of thousands of distinct ASL signs, along with the annotations of those sequences.

According to~\cite{iotsecurity}, Internet of Things (IoT) has become the main standard of low-power lossy network having constrained resource. IoT represents a network of sensor(s) embedded devices with the capability to communicate with other devices and perform a specific task. In 2014, Gartner predicted the number of connected IoT devices will exceed 20 billions by 2020~\cite{Gartner14}. The r
\ devices~\cite{conf/ndss/2013,batteryfirmware,hddfirmware}. Cui et al.~\cite{conf/ndss/2013} showed a scenario of an attack on a remote printer through a modified firmware which resulting in covert network scanning, data ex-filtration, and malware propagation to other devices in the network. Miller~\cite{batteryfirmware} reverse engineered the original firmware and the firmware flashing process of a particular smart battery system, and installed a modified firmware to the corresponding system to control the operation of the corresponding smart battery and smart battery host. In 2015, Zetter~\cite{hddfirmware} reported a hacking tool created by spy group \textit{Equation} that is able to re-flash the firmware of a hard drive and replace it with a malicious firmware. These three attack scenarios show the importance to maintain the device's firmware up-to-date and to securely deliver the firmware to corresponding IoT devices.

Firmware is a certain set of data and basic commands installed in a IoT device to perform a specific task~\cite{Rashedul16,Gajjar:2017:MSC:3110857}. Every IoT device needs firmware to associate with other software to interpret the collected raw data into some meaningful context. The instruction in the firmware can be defined as set of programmed routines to handle different components in the associate IoT device.

In order to ensure the IoT device works properly and to reduce any existing vulnerabilities in the device, the device's firmware needs to be updated regularly. In addition, the vendor could add new functionalities and re-configure the corresponding device through the new firmware version. Generally, IoT devices have several characteristics as follow: low-power consumption, limited memory and computing capability. Furthermore, most of IoT devices could not establish direct connection to the Internet and require the assistance from gateway or router to perform the firmware update process.

In general, the device firmware is considered more secure compared to other software due to its proprietary nature~\cite{Byung16}. However, it is reported more than 750.000 consumer "smart" appliances including home router, fridge, air conditioner, television, etc. had been compromised and used as bot to spam emails and distribute phishing in 2014~\cite{ProofPoint14}. As these smart appliances are connected to the Internet through a home router, once the router is compromised then all the connected devices could be easily affected as well. 

Blockchain ~\cite{Satoshi} was originally created for peer-to-peer electronic cash payment transaction based on cryptography proof instead of trusting third party, such as Bank. Blockchain uses distributed ledger to record digital transactions, in which these ledgers are distributed on decentralized peer-to-peer network. Each transaction stored in the blockchain's ledger is immutable, in which once the transaction is recorded in the ledger then it could not be modified by any entity in the blockchain network. In addition, all the transaction history recorded in the blockchain ledger are publicly traceable. 

In the current blockchain technology, users could trace all transactions history recorded in the blockchain by synchronizing the block data to the latest version. This ability requires the user's device to be connected to the Internet and synchronize with the latest block. However, almost all IoT devices are unable to join the blockchain network due to their limitation in Internet connectivity and storage capacity. Therefore, this thesis proposes a secure and trusted peer-to-peer (p2p) firmware update framework for IoT environment.

In this thesis, IoT devices are categorized into two groups based on the device capability to connect with Internet. The first group is called online IoT device, in which the device belong to this category is able to connect to the Internet without the assistance of any intermediary devices such as router. Any IoT device that can only connect to the Internet through the assistance of intermediary device is belonged to the second group, called offline IoT device. As most of the existing firmware update mechanism targets the online IoT device, it is difficult to guarantee the integrity of installed firmware on offline IoT device during the offline firmware update. The firmware update mechanism proposed in this thesis is designed to perform on both online and offline IoT device through the use of skipchain technology. Detail information regarding the skipchain technology~\cite{Skipchain} will be explained in Chapter~\ref{cha:2_literature}.

The update mechanism in the proposed scenario begins when the device manufacturer publishes the information of new firmware inside a smart contract to the blockchain network. Then, the newly created smart contract is verified by other nodes in the blockchain network. In this scenario, each gateway is connected to the blockchain network as a passive node, able to check all the verified smart contracts, and manages one or more IoT devices. In the case that the gateway detects the new firmware update, it will check if there is any connected device that needs this update. The IoT device that requires the firmware update will download the binary through a given url from the metadata given by its vendor. 

The contribution of this thesis are as follow:\begin{enumerate}
	\item This thesis introduce a new design of skipchain-based firmware update framework for IoT devices, especially for device with limited Internet connection.
	\item A secure firmware update distribution protocol is designed to support the proposed framework and firmware update mechanism.
	\item The security strength of the proposed protocol design is evaluated against well-known attacks: man-in-the-middle (MITM) attacks, replay attacks, firmware modification attacks, impersonation attacks, and isolation attack.
\end{enumerate}

The remainder of this thesis is organized as follow: introduction to skipchain and existing firmware update mechanisms are described in Chapter~\ref{cha:2_literature}. The assumptions used in this thesis, overview of the skipchain technology, the proposed system architecture and protocol designs are explained in Chapter~\ref{cha:3_design}. Prototype implementation of the proposed firmware update framework is discussed in Chapter \ref{cha:4_prototypeImplementation}. Chapter \ref{cha:5_analysis} describes the security and performance analyses conducted on the proposed protocol. Finally, this thesis is concluded in Chapter \ref{cha:6_conclusions}.

% \begin{table}[t!]
%   \begin{center}
%     \caption{The relation of aggregation overhead between different techniques}
%     \label{tab:1_system}
%     \begin{tabular}{|c|c c c|}
%       \hline
%       & Space usage & Communication & Query \\
%       & of root aggregator & overhead & requirement \\
%       \hline
%       Traditional warehouse & $n$ & $O(n)$ & $O(n)$ \\
%       \hline
%       AM-FM sketch technique & $\log a$ & $O(\log n)$ &  $O(a\log n)$ \\
%       \hline
%       ``prototypical PHI query'' & $\log a$ & $O(\log n)$ & $O(\log n)$ \\
%       \hline
%       \end{tabular}
%   \end{center}
% \end{table}