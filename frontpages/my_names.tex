%
% This file is encoded in utf-8.
% v1.7
% Enter your thesis title, your name, and your personal information.
% If the title needs to use math symbols, please use \mbox{}.
%   For example: My Thesis Title \mbox{$\cal{H}_\infty$} and \mbox{Al$_x$Ga$_{1-x}$As}
% If your Chinese name contains only 2 characters, please put space in between.
% If you need to put space between name and name's title (Prof. , Dr. ), 
%   please use tilde '~' instead of normal space ' '.
% The default number of advisors used on this template is three (3),
%   feel free to remove any of them as necessary.
%
% 填入你的論文題目、姓名等資料
% 如果題目內有必須以數學模式表示的符號,請用 \mbox{} 包住數學模式,如下範例
% 如果中文名字是單名,與姓氏之間建議以全形空白填入,如下範例
% 英文名字中的稱謂,如 Prof. 以及 Dr.,其句點之後請以不斷行空白~代替一般空白,如下範例
% 如果你的指導教授沒有如預設的三位這麼多,則請把相對應的多餘教授的中文、英文名
%    的定義以空的大括號表示
%    如,\renewcommand\advisorCnameB{}
%          \renewcommand\advisorEnameB{}
%          \renewcommand\advisorCnameC{}
%          \renewcommand\advisorEnameC{}

% Thesis title (Chinese)
% 論文題目 (中文)
% It's not necessary for foreign students to include the Chinese title on the thesis
\renewcommand\cTitle{
%可驗證之安全系統的應用
}

% Thesis title (Chinese)
% 論文題目 (英文)
\renewcommand\eTitle{  
Video Transition Smoothing for ASL Dataset
}

% My full name (Chinese)
% 我的姓名 (中文)
% It's not necessary for foreign students to include the Chinese name
%\renewcommand\myCname{王大明}
\renewcommand\myCname{Yulia}

% My full name (English)
% 我的姓名 (英文)
\renewcommand\myEname{Yulia}

% My student ID
%我的學號
\renewcommand\myStudentID{M10609803}

% My adviser's full name (Chinese)
% 指導教授A的姓名 (中文)
\renewcommand\advisorCnameA{Prof. Chuan-Kai Yang}
\renewcommand\advisorCnameB{}
\renewcommand\advisorCnameC{}

% My adviser's full name (English)
% 指導教授A的姓名 (英文)
\renewcommand\advisorEnameA{Prof. Chuan-Kai Yang}
\renewcommand\advisorEnameB{}
\renewcommand\advisorEnameC{}

% Campus' name (Chinese)
% 校名 (中文)
\renewcommand\univCname{國立台灣科技大學}

% Campus' name (English)
% 校名 (英文)
\renewcommand\univEname{National Taiwan University of Science and Technology}

% Department's name (Chinese)
% 系所名 (中文)
\renewcommand\deptCname{資~訊~管~理~系}

% Department's name (English)
% 系所全名 (英文)
\renewcommand\fulldeptEname{Department of Information Management}

% Department's short name (English)
% 系所短名 (英文, 用於書名頁學位名領域)
\renewcommand\deptEname{Department of Information Management}

% College's name (English)
% 學院英文名 (如無,則以空的大括號表示)
\renewcommand\collEname{School of Management}

% Degree's name (Chinese)
% 學位名 (中文)
\renewcommand\degreeCname{碩士}
%\renewcommand\degreeCname{博士}

% Degree's name (English)
% 學位名 (英文)
\renewcommand\degreeEname{Master of Information Management}
%\renewcommand\degreeEname{Doctor}

% Oral exam year (Chinese, Taiwanese Year)
% 口試年份 (中文、民國)
\renewcommand\cYear{一百零八}

% Oral exam month (Chinese)
% 口試月份 (中文)
\renewcommand\cMonth{六} 

% Oral exam date (Chinese)
% 口試日份 (中文)
\renewcommand\cDay{二十一} 

% Oral exam year (Arabic number, Gregorian Year)
% 口試年份 (阿拉伯數字、西元)
\renewcommand\eYear{2019} 

% Oral exam month (English)
% 口試月份 (英文)
\renewcommand\eMonth{June}

% Campus' location
% 學校所在地 (英文)
\renewcommand\ePlace{Taipei, Taiwan}

% Graduation year
%畢業級別;用於書背列印;若無此需要可忽略
\newcommand\GraduationClass{108}

%%%%%%%%%%%%%%%%%%%%%%